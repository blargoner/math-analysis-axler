% Notes and exercises from Measure, Integration & Real Analysis
% By John Peloquin
\documentclass[letterpaper,12pt]{article}
\usepackage{amsmath,amssymb,amsthm,fourier,enumitem}

\newcommand{\Z}{\mathbf{Z}}
\newcommand{\Zp}{\Z^+}
\newcommand{\Q}{\mathbf{Q}}
\newcommand{\R}{\mathbf{R}}
\newcommand{\B}{\mathcal{B}}
\renewcommand{\S}{\mathcal{S}}
\newcommand{\T}{\mathcal{T}}

\newcommand{\union}{\cup}
\newcommand{\bigunion}{\bigcup}
\newcommand{\sect}{\cap}
\newcommand{\bigsect}{\bigcap}
\newcommand{\mult}{\cdot}

\DeclareMathOperator{\len}{\ell}

\newcommand{\abs}[1]{|#1|}
\renewcommand{\l}[1]{\len(#1)}
\newcommand{\m}[1]{|#1|}
\newcommand{\bigmeasure}[1]{\Bigl|#1\Bigr|}
\newcommand{\inv}[1]{#1^{-1}}
\newcommand{\closure}[1]{\overline{#1}}

% Theorems
\theoremstyle{definition}
\newtheorem*{exer}{Exercise}

\theoremstyle{remark}
\newtheorem*{rmk}{Remark}

\theoremstyle{plain}
\newtheorem*{slogan}{Slogan}

% Meta
\title{Notes and exercises from\\\emph{Measure, Integration \& Real Analysis}}
\author{John Peloquin}
\date{}

\begin{document}
\maketitle

\section*{Introduction}
This document contains notes and exercises from~\cite{axler}. A slogan is provided for each result; this is what the man in the infomercial would yell at you when he is selling you the result.

\section*{Chapter~2}
\subsection*{Section~A}

\begin{exer}[1]
If \(A,B\subset\R\) and \(\m{B}=0\), then \(\m{A\union B}=\m{A}\).
\end{exer}
\begin{slogan}
Sets of outer measure zero don't affect outer measure!
\end{slogan}
\begin{proof}
By monotonicity and subadditivity of outer measure,
\[\m{A}\le\m{A\union B}\le\m{A}+\m{B}=\m{A}\qedhere\]
\end{proof}

\begin{exer}[2]
If \(A\subset\R\) and \(t\in\R\), then \(\m{tA}=\abs{t}\m{A}\), where \(tA=\{\,ta\mid a\in A\,\}\).
\end{exer}
\begin{slogan}
Outer measure dilates!
\end{slogan}
\begin{proof}
If \(t=0\), then the result is trivial (where we assume \(0\mult\infty=0\)). If \(t>0\), then for \(b,c\in\R\) with \(b<c\), \(t(b,c)=(tb,tc)\) and so
\[\l{t(b,c)}=tc-tb=t(c-b)=t\l{(b,c)}\]
More generally if \(t\ne0\) and \(I\subset\R\) is an arbitrary open interval, then \(tI\)~is an open interval with \(\l{tI}=\abs{t}\l{I}\).

Fix \(\epsilon>0\). Let \(I_1,I_2,\ldots\) be a sequence of open intervals with \(A\subset\bigunion_{k=1}^{\infty}I_k\) and
\[\sum_{k=1}^{\infty}\l{I_k}\le\m{A}+\frac{\epsilon}{\abs{t}}\]
By the above, \(tI_1,tI_2,\ldots\) is a sequence of open intervals with \(tA\subset\bigunion_{k=1}^{\infty}tI_k\) and
\[\m{tA}\le\sum_{k=1}^{\infty}\l{tI_k}=\abs{t}\sum_{k=1}^{\infty}\l{I_k}\le\abs{t}\,\Bigl(\m{A}+\frac{\epsilon}{\abs{t}}\Bigr)=\abs{t}\m{A}+\epsilon\]
Since \(\epsilon\)~is arbitrary, it follows that
\[\m{tA}\le\abs{t}\m{A}\]
Substituting simultaneously \(1/t\) for~\(t\) and \(tA\) for~\(A\) yields
\[\abs{t}\m{A}\le\m{tA}\]
so \(\m{tA}=\abs{t}\m{A}\) as desired.
\end{proof}

\begin{exer}[3]
If \(A,B\subset\R\) and \(\m{A}<\infty\), then \(\m{B\setminus A}\ge\m{B}-\m{A}\).
\end{exer}
\begin{slogan}
The outer measure of a difference is at least the difference of the outer measures!
\end{slogan}
\begin{proof}
Since \(B\subset A\union(B\setminus A)\),
\[\m{B}\le\m{A}+\m{B\setminus A}\]
The result follows by subtracting \(\m{A}\) from both sides.
\end{proof}
\begin{rmk}
The hypothesis \(\m{A}<\infty\) is necessary since \(\infty-\infty\) is undefined.
\end{rmk}

\begin{exer}[6]
If \(a,b\in\R\) and \(a<b\), then
\[\m{(a,b)}=\m{[a,b)}=\m{(a,b]}=b-a\]
\end{exer}
\begin{slogan}
The outer measure of any interval is its length!
\end{slogan}
\begin{proof}
For example, \([a,b]=(a,b)\union\{a,b\}\) and \(\m{\{a,b\}}=0\), so
\[\m{(a,b)}=\m{[a,b]}=b-a\qedhere\]
\end{proof}

\begin{exer}[7]
If \(a,b,c,d\in\R\) with \(a<b\) and \(c<d\), then
\[\m{(a,b)\union(c,d)}=(b-a)+(d-c)\quad\text{if and only if}\quad(a,b)\sect(c,d)=\emptyset\]
\end{exer}
\begin{slogan}
Outer measure is finitely additive on intervals!
\end{slogan}
\begin{proof}
If \((a,b)\sect(c,d)=\emptyset\), then a generalization of the proof of~2.14 shows that
\[\m{(a,b)\union(c,d)}=\m{[a,b]\union[c,d]}=(b-a)+(d-c)\]
If \((a,b)\sect(c,d)\ne\emptyset\), we may assume \(a\le c<b\). If \(b\le d\), then \((a,b)\union(c,d)=(a,d)\) and
\[\m{(a,d)}=d-a<(b-a)+(d-c)\]
If \(d<b\), then \((a,b)\union(c,d)=(a,b)\) and
\[\m{(a,b)}=b-a<(b-a)+(d-c)\qedhere\]
\end{proof}
\begin{rmk}
This proof generalizes to an arbitrary finite number of intervals.
\end{rmk}

\begin{exer}[10]
\(\m{[0,1]\setminus\Q}=1\)
\begin{slogan}
Almost every number in the interval~\([0,1]\) is irrational!
\end{slogan}
\end{exer}
\begin{proof}
By Exercise~3, since \(\m{[0,1]}=1\) and \(\m{\Q}=0\).
\end{proof}

\begin{exer}[11]
If \(I_1,I_2,\ldots\) is a disjoint sequence of open intervals, then
\[\bigmeasure{\bigunion_{k=1}^{\infty}I_k}=\sum_{k=1}^{\infty}\l{I_k}\]
\end{exer}
\begin{slogan}
Outer measure is additive on intervals!
\end{slogan}
\begin{proof}
If any of the intervals are unbounded, then the result is trivial. If all of the intervals are bounded, then we may also assume that they are nonempty and the result follows from Exercise~7 since for all~\(n\),
\[\sum_{k=1}^n\l{I_k}=\bigmeasure{\bigunion_{k=1}^n I_k}\le\bigmeasure{\bigunion_{k=1}^{\infty}I_k}\]
so
\[\sum_{k=1}^{\infty}\l{I_k}\le\bigmeasure{\bigunion_{k=1}^{\infty}I_k}\le\sum_{k=1}^{\infty}\l{I_k}\qedhere\]
\end{proof}

\begin{exer}[13]
For any \(\epsilon>0\), there exists \(F\subset[0,1]\setminus\Q\) closed in~\(\R\) with \(\m{F}\ge1-\epsilon\).
\end{exer}
\begin{slogan}
The interval~\([0,1]\) is well approximated by closed subsets of irrationals!
\end{slogan}
\begin{proof}
Let \(r_1,r_2,\ldots\) be an enumeration of \([0,1]\sect\Q\) and define
\[F=[0,1]\setminus\bigunion_{k=1}^{\infty}\Bigl(r_k-\frac{\epsilon}{2^{k+1}},r_k+\frac{\epsilon}{2^{k+1}}\Bigr)\]
Clearly \(F\subset[0,1]\setminus\Q\), \(F\)~is closed in~\(\R\), and \(\m{F}\ge1-\epsilon\) by Exercise~3 since
\[\bigmeasure{\bigunion_{k=1}^{\infty}\Bigl(r_k-\frac{\epsilon}{2^{k+1}},r_k+\frac{\epsilon}{2^{k+1}}\Bigr)}\le\sum_{k=1}^{\infty}\frac{\epsilon}{2^k}=\epsilon\qedhere\]
\end{proof}

% References
\newpage
\begin{thebibliography}{0}
\bibitem{axler} Axler, S. \textit{Measure, Integration \& Real Analysis.} Springer, 2020.
\end{thebibliography}
\end{document}