% Notes and exercises from Measure, Integration & Real Analysis
% By John Peloquin
\documentclass[letterpaper,12pt]{article}
\usepackage{amsmath,amssymb,amsthm,fourier,enumitem}

\newcommand{\Z}{\mathbf{Z}}
\newcommand{\Zp}{\Z^+}
\newcommand{\Q}{\mathbf{Q}}
\newcommand{\R}{\mathbf{R}}
\newcommand{\B}{\mathcal{B}}
\renewcommand{\S}{\mathcal{S}}
\newcommand{\T}{\mathcal{T}}

\newcommand{\union}{\cup}
\newcommand{\bigunion}{\bigcup}
\newcommand{\sect}{\cap}
\newcommand{\bigsect}{\bigcap}
\newcommand{\mult}{\cdot}

\DeclareMathOperator{\len}{\ell}

\newcommand{\abs}[1]{|#1|}
\renewcommand{\l}[1]{\len(#1)}
\newcommand{\m}[1]{|#1|}
\newcommand{\bigmeasure}[1]{\Bigl|#1\Bigr|}
\newcommand{\inv}[1]{#1^{-1}}
\newcommand{\closure}[1]{\overline{#1}}

% Theorems
\theoremstyle{definition}
\newtheorem*{exer}{Exercise}

\theoremstyle{remark}
\newtheorem*{rmk}{Remark}

\theoremstyle{plain}
\newtheorem*{slogan}{Slogan}

% Meta
\title{Notes and exercises from\\\emph{Measure, Integration \& Real Analysis}}
\author{John Peloquin}
\date{}

\begin{document}
\maketitle

\section*{Introduction}
This document contains notes and exercises from~\cite{axler}. A slogan is provided for each result; this is what the man in the infomercial would yell at you when he is selling you the result.

\section*{Chapter~2}
\subsection*{Section~A}

\begin{exer}[1]
If \(A,B\subset\R\) and \(\m{B}=0\), then \(\m{A\union B}=\m{A}\).
\end{exer}
\begin{slogan}
Sets of outer measure zero don't affect outer measure!
\end{slogan}
\begin{proof}
By monotonicity and subadditivity of outer measure,
\[\m{A}\le\m{A\union B}\le\m{A}+\m{B}=\m{A}\qedhere\]
\end{proof}

\begin{exer}[2]
If \(A\subset\R\) and \(t\in\R\), then \(\m{tA}=\abs{t}\m{A}\), where \(tA=\{\,ta\mid a\in A\,\}\).
\end{exer}
\begin{slogan}
Outer measure dilates!
\end{slogan}
\begin{proof}
If \(t=0\), then the result is trivial (where we assume \(0\mult\infty=0\)). If \(t>0\), then for \(b,c\in\R\) with \(b<c\), \(t(b,c)=(tb,tc)\) and so
\[\l{t(b,c)}=tc-tb=t(c-b)=t\l{(b,c)}\]
More generally if \(t\ne0\) and \(I\subset\R\) is an arbitrary open interval, then \(tI\)~is an open interval with \(\l{tI}=\abs{t}\l{I}\).

Fix \(\epsilon>0\). Let \(I_1,I_2,\ldots\) be a sequence of open intervals with \(A\subset\bigunion_{k=1}^{\infty}I_k\) and
\[\sum_{k=1}^{\infty}\l{I_k}\le\m{A}+\frac{\epsilon}{\abs{t}}\]
By the above, \(tI_1,tI_2,\ldots\) is a sequence of open intervals with \(tA\subset\bigunion_{k=1}^{\infty}tI_k\) and
\[\m{tA}\le\sum_{k=1}^{\infty}\l{tI_k}=\abs{t}\sum_{k=1}^{\infty}\l{I_k}\le\abs{t}\,\Bigl(\m{A}+\frac{\epsilon}{\abs{t}}\Bigr)=\abs{t}\m{A}+\epsilon\]
Since \(\epsilon\)~is arbitrary, it follows that
\[\m{tA}\le\abs{t}\m{A}\]
Substituting simultaneously \(1/t\) for~\(t\) and \(tA\) for~\(A\) yields
\[\abs{t}\m{A}\le\m{tA}\]
so \(\m{tA}=\abs{t}\m{A}\) as desired.
\end{proof}

\begin{exer}[3]
If \(A,B\subset\R\) and \(\m{A}<\infty\), then \(\m{B\setminus A}\ge\m{B}-\m{A}\).
\end{exer}
\begin{slogan}
The outer measure of a difference is at least the difference of the outer measures!
\end{slogan}
\begin{proof}
Since \(B\subset A\union(B\setminus A)\),
\[\m{B}\le\m{A}+\m{B\setminus A}\]
The result follows by subtracting \(\m{A}\) from both sides.
\end{proof}
\begin{rmk}
The hypothesis \(\m{A}<\infty\) is necessary since \(\infty-\infty\) is undefined.
\end{rmk}

\begin{exer}[6]
If \(a,b\in\R\) and \(a<b\), then
\[\m{(a,b)}=\m{[a,b)}=\m{(a,b]}=b-a\]
\end{exer}
\begin{slogan}
The outer measure of any interval is its length!
\end{slogan}
\begin{proof}
For example, \([a,b]=(a,b)\union\{a,b\}\) and \(\m{\{a,b\}}=0\), so
\[\m{(a,b)}=\m{[a,b]}=b-a\qedhere\]
\end{proof}

\begin{exer}[7]
If \(a,b,c,d\in\R\) with \(a<b\) and \(c<d\), then
\[\m{(a,b)\union(c,d)}=(b-a)+(d-c)\quad\text{if and only if}\quad(a,b)\sect(c,d)=\emptyset\]
\end{exer}
\begin{slogan}
Outer measure is finitely additive on intervals!
\end{slogan}
\begin{proof}
If \((a,b)\sect(c,d)=\emptyset\), then a generalization of the proof of~2.14 shows that
\[\m{(a,b)\union(c,d)}=\m{[a,b]\union[c,d]}=(b-a)+(d-c)\]
If \((a,b)\sect(c,d)\ne\emptyset\), we may assume \(a\le c<b\). If \(b\le d\), then \((a,b)\union(c,d)=(a,d)\) and
\[\m{(a,d)}=d-a<(b-a)+(d-c)\]
If \(d<b\), then \((a,b)\union(c,d)=(a,b)\) and
\[\m{(a,b)}=b-a<(b-a)+(d-c)\qedhere\]
\end{proof}
\begin{rmk}
This proof generalizes to an arbitrary finite number of intervals.
\end{rmk}

\begin{exer}[10]
\(\m{[0,1]\setminus\Q}=1\)
\begin{slogan}
Almost every number in the interval~\([0,1]\) is irrational!
\end{slogan}
\end{exer}
\begin{proof}
By Exercise~3, since \(\m{[0,1]}=1\) and \(\m{\Q}=0\).
\end{proof}

\begin{exer}[11]
If \(I_1,I_2,\ldots\) is a disjoint sequence of open intervals, then
\[\bigmeasure{\bigunion_{k=1}^{\infty}I_k}=\sum_{k=1}^{\infty}\l{I_k}\]
\end{exer}
\begin{slogan}
Outer measure is additive on intervals!
\end{slogan}
\begin{proof}
If any of the intervals are unbounded, then the result is trivial. If all of the intervals are bounded, then we may also assume that they are nonempty and the result follows from Exercise~7 since for all~\(n\),
\[\sum_{k=1}^n\l{I_k}=\bigmeasure{\bigunion_{k=1}^n I_k}\le\bigmeasure{\bigunion_{k=1}^{\infty}I_k}\]
so
\[\sum_{k=1}^{\infty}\l{I_k}\le\bigmeasure{\bigunion_{k=1}^{\infty}I_k}\le\sum_{k=1}^{\infty}\l{I_k}\qedhere\]
\end{proof}

\begin{exer}[13]
For any \(\epsilon>0\), there exists \(F\subset[0,1]\setminus\Q\) closed in~\(\R\) with \(\m{F}\ge1-\epsilon\).
\end{exer}
\begin{slogan}
The interval~\([0,1]\) is well approximated by closed subsets of irrationals!
\end{slogan}
\begin{proof}
Let \(r_1,r_2,\ldots\) be an enumeration of \([0,1]\sect\Q\) and define
\[F=[0,1]\setminus\bigunion_{k=1}^{\infty}\Bigl(r_k-\frac{\epsilon}{2^{k+1}},r_k+\frac{\epsilon}{2^{k+1}}\Bigr)\]
Clearly \(F\subset[0,1]\setminus\Q\), \(F\)~is closed in~\(\R\), and \(\m{F}\ge1-\epsilon\) by Exercise~3 since
\[\bigmeasure{\bigunion_{k=1}^{\infty}\Bigl(r_k-\frac{\epsilon}{2^{k+1}},r_k+\frac{\epsilon}{2^{k+1}}\Bigr)}\le\sum_{k=1}^{\infty}\frac{\epsilon}{2^k}=\epsilon\qedhere\]
\end{proof}

\subsection*{Section~B}
\begin{exer}[3]
The \(\sigma\)-algebra \(\S\) on~\(\R\) generated by the intervals \((r,s]\) with \(r,s\in\Q\) is the collection~\(\B\) of Borel subsets of~\(\R\).
\end{exer}
\begin{slogan}
The Borel sets are generated by intervals with rational endpoints!
\end{slogan}
\begin{proof}
We have \(\S\subset\B\) since \(\B\)~is a \(\sigma\)-algebra on~\(\R\) with \((r,s]\in\B\) for all \(r,s\in\Q\) (2.30). On the other hand, clearly any open subset of~\(\R\) is a union of intervals of the form \((r,s]\) with \(r,s\in\Q\) by density of~\(\Q\) in~\(\R\), and any such union is countable by countability of~\(\Q\). Therefore \(\S\)~contains the open subsets of~\(\R\), so \(\B\subset\S\).
\end{proof}

\begin{exer}[7]
The collection~\(\B\) of Borel subsets of~\(\R\) is translation invariant.
\end{exer}
\begin{slogan}
A translation of a Borel set is a Borel set!
\end{slogan}
\begin{proof}
Fix \(t\in\R\) and let
\[\S=\{\,B\in\B\mid t+B\in\B\,\}\]
We claim that \(\S\)~is a \(\sigma\)-algebra on~\(\R\) containing the open intervals, so \(\S=\B\). Clearly \(\S\)~contains the open intervals (including the empty interval). If \(B\in\S\), then \(\R\setminus B\in\B\) and
\[t+\R\setminus B=\R\setminus(t+B)\in\B\]
by assumption, so \(\R\setminus B\in\S\). If \(B_1,B_2,\ldots\in\S\), then \(\bigunion_{k=1}^{\infty} B_k\in\B\) and
\[t+\bigunion_{k=1}^{\infty}B_k=\bigunion_{k=1}^{\infty}(t+B_k)\in\B\]
by assumption, so \(\bigunion_{k=1}^{\infty}B_k\in\S\).
\end{proof}

\begin{exer}[9]
Let \(\S=\{\emptyset,\R\}\) be the trivial \(\sigma\)-algebra on~\(\R\) and define \(f:\R\to\R\) by
\[f(x)=\begin{cases}
-1&\text{if }x<0\\
1&\text{if }x\ge 0
\end{cases}\]
Then \(f\)~is not \(\S\)-measurable since it is not constant, but \(\abs{f}=1\) is (2.36).
\end{exer}
\begin{slogan}
A function with measurable absolute value need not be measurable!
\end{slogan}

\begin{exer}[12]
Let \(f:\R\to\R\).
\begin{enumerate}[itemsep=0pt]
\item[(a)] For \(k\in\Zp\), let
\[G_k=\{\,a\in\R\mid[\exists\delta>0]\ [\forall b,c\in(a-\delta,a+\delta)]\ \abs{f(b)-f(c)}<\tfrac{1}{k}\,\}\]
\(G_k\)~is open in~\(\R\).
\item[(b)] The set of points of continuity of~\(f\) is~\(\bigsect_{k=1}^{\infty}G_k\).
\item[(c)] The set of points of continuity of~\(f\) is Borel.
\end{enumerate}
\end{exer}
\begin{slogan}
A real-valued function of a real variable is continuous on a Borel set!
\end{slogan}
\begin{proof}
For~(a), if \(a\in G_k\) and \(\delta>0\) is chosen as in the definition of~\(G_k\), then \((a-\delta,a+\delta)\subset G_k\) since \((a-\delta,a+\delta)\) is open. Therefore \(G_k\)~is open.

For~(b), if \(f\)~is continuous at \(a\in\R\), then for any \(k\in\Zp\) there is \(\delta>0\) with
\[\abs{f(b)-f(a)}<\frac{1}{2k}\]
for all \(b\in(a-\delta,a+\delta)\). Now for all \(b,c\in(a-\delta,a+\delta)\),
\[\abs{f(b)-f(c)}\le\abs{f(b)-f(a)}+\abs{f(a)-f(c)}<\frac{1}{2k}+\frac{1}{2k}=\frac{1}{k}\]
Therefore \(a\in G_k\). Since \(k\)~was arbitrary, \(a\in\bigsect_{k=1}^{\infty} G_k\). Conversely if \(a\in\bigsect_{k=1}^{\infty} G_k\), then for any \(\epsilon>0\) there is \(k\in\Zp\) with \(1/k\le\epsilon\) and since \(a\in G_k\) there is \(\delta>0\) with
\[\abs{f(b)-f(c)}<\frac{1}{k}\le\epsilon\]
for all \(b,c\in(a-\delta,a+\delta)\), in particular for \(c=a\). Thus \(f\)~is continuous at~\(a\).

Now (c)~follows from (a) and~(b) since a countable intersection of open sets is Borel (2.25).
\end{proof}

\begin{exer}[13]
Let \((X,\S)\) be a measurable space. If \(c_1,\ldots,c_n\) are distinct nonzero real numbers and \(E_1,\ldots,E_n\) are disjoint subsets of~\(X\), then
\[\chi=c_1\chi_{E_1}+\cdots+c_n\chi_{E_n}\]
is \(\S\)-measurable if and only if \(E_1,\ldots,E_n\in\S\).
\end{exer}
\begin{slogan}
A linear combination of characteristic functions is measurable if and only if the sets are measurable!
\end{slogan}
\begin{proof}
By assumption, \(\inv{\chi}(c_k)=E_k\) for all~\(k\). Since \(\{c_k\}\)~is Borel (2.30), it follows that \(E_1,\ldots,E_n\in\S\) if \(\chi\)~is \(\S\)-measurable.

Conversely if \(E_1,\ldots,E_n\in\S\), then \(\chi_{E_1},\ldots,\chi_{E_n}\) are \(\S\)-measurable (2.38), so \(c_1\chi_{E_1},\ldots,c_n\chi_{E_n}\) are \(\S\)-measurable (2.44), so \(\chi\)~is \(\S\)-measurable (2.46). (This is true even without the other assumptions on the \(c_k\) and~\(E_k\).)
\end{proof}

\begin{exer}[14]
Let \(f_1,f_2,\ldots:X\to\R\) and
\[L=\{\,x\in X\mid\lim_{k\to\infty}f_k(x)\text{ exists}\,\}\]
\begin{enumerate}[itemsep=0pt]
\item[(a)] \(\displaystyle L=\bigsect_{n=1}^{\infty}\bigunion_{j=1}^{\infty}\bigsect_{k=j}^{\infty}\inv{(f_j-f_k)}((-\tfrac{1}{n},\tfrac{1}{n}))\)
\item[(b)] If \(\S\)~is a \(\sigma\)-algebra on~\(X\) and each \(f_k\)~is \(\S\)-measurable, then \(L\in\S\).
\end{enumerate}
\end{exer}
\begin{slogan}
A sequence of measurable real-valued functions converges pointwise on a measurable set!
\end{slogan}
\begin{proof}
Part~(a) follows from the Cauchy criterion for convergence of sequences in~\(\R\), and (b)~follows from (a) and~2.46.
\end{proof}

\begin{exer}[15]
Let \(X\)~be a set and \(X_1,X_2,\ldots\) a sequence of disjoint subsets of~\(X\) with \(\bigunion_{k=1}^{\infty}X_k=X\). Define
\[\S=\{\,\bigunion_{k\in K}X_k\mid K\subset\Zp\,\}\]
\begin{enumerate}[itemsep=0pt]
\item[(a)] \(\S\)~is a \(\sigma\)-algebra on~\(X\).
\item[(b)] A function \(f:X\to\R\) is \(\S\)-measurable if and only if \(f\)~is constant on~\(X_k\) for all \(k\in\Zp\).
\end{enumerate}
\end{exer}
\begin{slogan}
A real-valued function is measurable over a countable partition if and only if it's constant on each set in the partition!
\end{slogan}
\begin{proof}
For~(a), \(K=\emptyset\) yields \(\emptyset\in\S\). If \(E\in\S\), then \(E=\bigunion_{k\in K}X_k\) for some \(K\subset\Zp\), and for \(K'=\Zp\setminus K\),
\[X\setminus E=\bigunion_{k\in K'}X_k\in\S\]
If \(E_1,E_2,\ldots\in\S\), then for each \(j\in\Zp\) we have \(E_j=\bigunion_{k\in K_j}X_k\) for some \(K_j\subset\Zp\). Now for \(K'=\bigunion_{j=1}^{\infty}K_j\),
\[\bigunion_{j=1}^{\infty}E_j=\bigunion_{k\in K'}X_k\in\S\]
For~(b), if \(f\)~is constant on each~\(X_k\), then for \(Y\subset\R\) and
\[K=\{\,k\in\Zp\mid f(X_k)\subset Y\,\}\]
we have
\[\inv{f}(Y)=\bigunion_{k\in K}X_k\in\S\]
so \(f\)~is \(\S\)-measurable. Conversely if \(f\)~is not constant on~\(X_k\) and \(y\in f(X_k)\), then \(\{y\}\)~is Borel but \(\inv{f}(y)\not\in\S\), so \(f\)~is not \(\S\)-measurable.
\end{proof}
\begin{rmk}
Exercise~1 is a special case of this exercise.
\end{rmk}

\begin{exer}[17]
Let \(B\subset\R\) be Borel. If \(f:B\to\R\) and
\[D=\{\,x\in B\mid f\text{ is not continuous at }x\,\}\]
is countable, then \(f\)~is Borel measurable.
\end{exer}
\begin{slogan}
A real-valued function of a real variable continuous at all but countably many points is Borel!
\end{slogan}
\begin{proof}
By a generalization of the proof of~2.41. If \(a\in\R\) and \(x\in B\) with \(f(x)>a\), then either \(x\in D\) or else by continuity of~\(f\) at~\(x\) there is \(\delta_x>0\) with \(f(y)>a\) for all \(y\in(x-\delta_x,x+\delta_x)\sect B\). Therefore \(X=\inv{f}((a,\infty))\) satisfies
\[X=(X\sect D)\union\bigl(B\sect\!\!\!\bigunion_{x\in X\setminus D}\!\!\!(x-\delta_x,x+\delta_x)\bigr)\]
Now \(X\sect D\)~is countable and hence Borel (2.30), so it follows that \(X\)~is Borel and therefore \(f\)~is Borel measurable (2.39).
\end{proof}

\begin{exer}[18]
If \(f:\R\to\R\) is differentiable, then \(f'\)~is Borel measurable.
\end{exer}
\begin{slogan}
Derivatives are Borel!
\end{slogan}
\begin{proof}
We have
\begin{align*}
f'(x)&=\lim_{h\to 0}\frac{f(x+h)-f(x)}{h}\\
	&=\lim_{k\to\infty}\frac{f(x+1/k)-f(x)}{1/k}\\
	&=\lim_{k\to\infty}k[f(x+1/k)-f(x)]
\end{align*}
Define \(g_k(x)=k[f(x+1/k)-f(x)]\). Then clearly \(g_k\)~is Borel measurable since it is continuous (2.41), so \(f'=\lim_{k\to\infty}g_k\) is Borel measurable (2.48).
\end{proof}

\begin{exer}[20]
Let \((X,\S)\)~be a measurable space. If \(f,g:X\to\R\) are \(\S\)-measurable and \(f>0\), then \(f^g\)~is \(\S\)-measurable.
\end{exer}
\begin{slogan}
Powers of measurable functions are measurable!
\end{slogan}
\begin{proof}
We have
\[f(x)^{g(x)}=\exp(\log f(x))^{g(x)}=\exp(g(x)\mult\log f(x))\]
Now \(\exp\) and~\(\log\) are continuous and hence Borel (2.41), so it follows that \(f^g\)~is measurable (2.44, 2.46).
\end{proof}

\begin{exer}[22]
If \(X\subset\R\) and \(f:X\to\R\) is increasing, then \(f\)~is continuous at all but countably many points of~\(X\).
\end{exer}
\begin{slogan}
An increasing function is continuous at all but countably many points!
\end{slogan}
\begin{proof}
For \(x\in X\), define
\[f(x^-)=\begin{cases}
\sup\{\,f(z)\mid z\in X\sect(-\infty,x)\,\}&\text{if }x\in\closure{X\sect(-\infty,x)}\\
f(x)&\text{otherwise}
\end{cases}\]
and similarly define~\(f(x^+)\). Since \(f\)~is increasing, \(f(x^-)\le f(x)\le f(x^+)\) and the inequalities are equalities if and only if \(f\)~is continuous at~\(x\). Moreover if \(x<y\), then \(f(x^+)\le f(y^-)\). Therefore the discontinuities of~\(f\) yield disjoint nonempty open intervals of the form \((f(x^-),f(x^+))\) in~\(\R\), of which there are only countably many by density and countability of~\(\Q\) in~\(\R\).
\end{proof}

\begin{exer}[23]
If \(f:\R\to\R\) is strictly increasing (but not necessarily continuous), then \(\inv{f}:f(\R)\to\R\) is continuous.
\end{exer}
\begin{slogan}
The inverse of a strictly increasing function is continuous!
\end{slogan}
\begin{proof}
For \(a,b\in\R\) with \(a<b\), we have
\[f((a,b))=(f(a),f(b))\sect f(\R)\]
Therefore \(f=\inv{(\inv{f})}\) is open (2.33), so \(\inv{f}\)~is continuous.
\end{proof}

\begin{exer}[27]
Let \(\S=\{\emptyset,\R\}\) and define \(f:\R\to[-\infty,\infty]\) by
\[f(x)=\begin{cases}
-\infty&\text{if }x<0\\
\infty&\text{if }x\ge0
\end{cases}\]
Then for every \(a\in\R\), \(\inv{f}((a,\infty))=\emptyset\in\S\), but \(f\)~is not \(\S\)-measurable since it is not constant (for example \(\inv{f}(\{\infty\})=[0,\infty)\not\in\S\)).
\end{exer}
\begin{slogan}
For extended real-valued functions, infinity can't be ignored!
\end{slogan}

\begin{exer}[28]
If \(f:B\to\R\) is Borel measurable and \(g:\R\to\R\) is defined by
\[g(x)=\begin{cases}
f(x)&\text{if }x\in B\\
0&\text{if }x\not\in B
\end{cases}\]
then \(g\)~is Borel measurable.
\end{exer}
\begin{slogan}
Borel measurable functions can be extended to be total!
\end{slogan}
\begin{proof}
Since \(B=\inv{f}(\R)\) is Borel, so is \(\R\setminus B\). For \(a\in\R\),
\[\inv{g}((a,\infty))=\begin{cases}
\inv{f}((a,\infty))&\text{if }a\ge0\\
\inv{f}((a,\infty))\union(\R\setminus B)&\text{if }a<0
\end{cases}\]
so \(\inv{g}((a,\infty))\)~is Borel and \(g\)~is Borel measurable (2.39).
\end{proof}

% References
\newpage
\begin{thebibliography}{0}
\bibitem{axler} Axler, S. \textit{Measure, Integration \& Real Analysis.} Springer, 2020.
\end{thebibliography}
\end{document}